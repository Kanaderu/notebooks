\documentclass[convert={density=300,size=350x200,outext=.png}]{standalone}
% \documentclass{standalone}
\usepackage{tikz}
\usetikzlibrary{shapes, arrows}

\begin{document}
\tikzstyle{block} = [draw, rectangle, fill=blue!20,
  minimum height=3em, minimum width=3em]
\tikzstyle{multiply} = [draw, circle, fill=blue!20, minimum size=2em]
\tikzstyle{point} = [coordinate]
\tikzstyle{input} = [coordinate]
\tikzstyle{output} = [coordinate]
\begin{tikzpicture}[auto, node distance=2cm, >=latex']
  \matrix[row sep=2em, column sep=3.5em] {
    \node [input] (input) {}; & 
    \node [point] (sum) {} ; &
    \node [block] (soma) {rLIF};\\ 
    & & \node [multiply] (af) {$\alpha_f$}; & \node[block] (hf) {$h_f(t)$}; \\
  };
  \draw (input) -- node {$u_{in}(t)$} (sum);
  \draw [->] (sum) -- node {$u_{net}(t)$} (soma);
%   \draw (B) -- (syn);
%   \draw [->] (sum) -- (syn);
%   \draw (syn) --  node {$x(t)$} (neurons);
  \draw [->] (soma) -| (hf);
  \draw [->] (hf) -- (af);
  \draw [->] (af) -| node {$u_{f}(t)$} (sum);
%   \draw [->] (n1) -- node {$d_1\eta_1$} (spk ens);
%   \draw [->] (n2) -- node {$d_2\eta_2$} (neurons);
\end{tikzpicture} 
\end{document}
